\subsection{Hypotheses}
Before describing our model, we discuss our hypotheses for how self-driving, communicating cars will affect traffic flow.  We take on good faith that by the time self-driving cars are road-ready, they will (1) consistently have shorter event-response time (such as the time it takes to suddenly brake) than a human and (2) have better situational awareness (knowledge of traffic patters well-ahead and well-behind them) than a human.  In addition, we do not account for the changes in traffic engendered by inclement weather, although according to an article in \textit{The New York Times}, self-driving cars do not yet have the capacity to reliably navigate rain and snow conditions \cite{Boudette20165Headaches}.   Based on these qualitative assumptions, we hypothesize that increasing frequencies of self-driving cars will:

\begin{enumerate}
\item reduce the travel time of going from A to B on busy roadways
\item decrease the variability of traffic flux across road segments 
\item reduce the frequency of phantom traffic jams 
\end{enumerate}