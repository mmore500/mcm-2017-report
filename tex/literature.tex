\subsection{Literature Search}

Mathematical models of traffic can generally be classified as either discrete or continuous.  In an extensive analysis of traffic models, one author estimates  that in the past 50 years, researchers across many fields have suggested nearly 100 different ways of modeling traffic. \cite{Helbing2001TrafficSystems} While a comprehensive review of each of these models is impossible, the ones we found most useful merit some discussion.  We began by reproducing Kai Nagel and Michael Schreckenberg's cellular automata model of single-lane traffic \cite{Nagel1992ATraffic}, which gave us insight into the traffic modeling process and enabled us to collect traffic flux data for generic traffic flow. We applied these results to a model that we independently developed based on the guiding principles presented in Fowkes and Mahony's \textit{An Introduction to Mathematical Modeling}. 

In the course of our research, we also found Kachroo and Sastry's \textit{Traffic Flow Theory} text \cite{KachrooTrafficFramework} to be a useful resource for evaluating the results of our model.  Although their various PDE models did not lend themselves to adaptation to account for self-driving cars, we were able to reproduce the flux-versus-traffic-density curves for the Greenshield and and Greenberg models that appear in Kachroo and Sastry's work, which we used as a metric for success of our fundamental traffic model.  It would be interesting to compare our results to a PDE model that incorporates self-driving cars.

One of the most prevalent traffic phenomena that appears in research concerns ``phantom traffic jams,''---traffic back-ups that happen even when nothing obstructs traffic flow---which \textbf{occur when human drivers act imperfectly}.  In particular, Helbing identifies \textit{overcorrection} and \textit{chain reaction} as two ways that mathematical traffic models successfully simulate these phantom traffic jams \cite{Helbing2001TrafficSystems}.  We used this information to determine reasonable ways of modeling self-driving cars as distinct from human-driven cars.

