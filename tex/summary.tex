{
\centering 
\textbf{Summary Sheet} \par
}

We present a model of traffic in the greater Seattle area to understand how an increasing frequency of self-driving cars will change traffic dynamics in the area.  We apply a \textbf{two-component micro/macro traffic simulation} to data for portions of Interstates 5, 90, 405, and State Route 520 to consider the impact of autonomous vehicles on regional traffic flow. We consider 0\%, 10\%, 50\%, and 90\% autonomous traffic.

Our micro model is designed to make predictions about the impact of self-driving vehicles on fundamental traffic dynamics and employs a \textbf{cellular automata approach}, inpsired by Nagel and Schrekenberg \cite{Nagel1992ATraffic}, to model interactions between a number of independent vehicles on a road. In this simulation, vehicles exhibit simple following behavior and experience occasional random deceleration events. We introduce a distinction between self-driving and human-driven cars, where autonomous vehicles exhibit \textbf{more uniform cruising speed} compared to human drivers and can \textbf{follow safely at a much closer distance}, compared to human drivers. 

Using this micro-level simulation, we predict a relation between traffic speed and traffic density for traffic with a varying composition of autonomous vehicles. Our macro model employs a system of ordinary differential equations to investigate the flow of traffic between segments of road in the region of study. We assess the impact of self-driving traffic composition on performance of the regional highway network at peak and average traffic loads, measuring trip times along each major highway and between a representative set of regional destinations. The travel time predictions of the macro model are \textbf{compared to archived travel time data} from the the Washington State Department of Transportation (WSDOT).

These models, in conjunction, facilitate insightful study of how different percentages of self-driving cars on the motorways change traffic flow under heavy and light traffic conditions. The quantitative accuracy of our macro model is observed to decline significantly with increasing traffic loads. Nevertheless, the results of our study demonstrate clear qualitative trends that inform our recommendations. Although our macro model does not make quantitatively accurate predictions, we observe a trend indicating that \textbf{at high traffic densities, traffic delays decrease with increasing percentages of self-driving cars} on the road. 

Analysis of our micro model reveals that assigning traffic lanes for the exclusive use of autonomous vehicles can be a boon to traffic flow efficiency. When the concentration of self-driving cars rises to above 5\%, our micro model predicts that it becomes advantageous to \textbf{implement at least one ``self-driving-car only" lane} in roads with 3 or more lanes. Under some circumstances, this strategy has the potential to result in \textbf{reduced travel delays for human-driven and autonomously controlled vehicles alike.}

Further study of the potential effects of autonomous vehicles on urban traffic is necessary. Our analysis does not address important questions related to self-driving cars that could have huge impacts on traffic conditions in the Seattle metropolitan area, such as the impact of autonomous technology on the total number of vehicles on the road and the temporal distribution of traffic loads.  Further modeling refinement is necessary at the macro level to make rigorous quantitative predictions about the impact of autonomous vehicles on regional traffic patterns; our model excludes many potentially significant factors such as accidents, road-closings, inclement weather, and aggressive drivers, to name but a few. Nonetheless, our model rigorously demonstrates that including significant proportion of self-driving cars on the road impacts traffic dynamics and suggests that such changes in traffic composition might meaningfully, although not totally, reduce the traffic congestion experienced by Seattle motorists.
\newpage