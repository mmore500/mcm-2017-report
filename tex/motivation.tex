\subsection{Motivation}
With the continual rise of urbanization, the logistics of driving in metropolitan areas are an ever-pressing concern. In the Seattle metropolitan area, traffic jams, which occur regularly during weekday rush hours, are of particular concern. In 2015, Seattle was ranked 7th in the nation for the worst traffic conditions for auto commuters \cite{Schrank20152015Scorecard}. The cost of traffic is no mere inconvenience -- it has a high economic, environmental, and quality of life price. Seattle's auto commuters are estimated to lose 63 hours a year to traffic delays, which are estimated to cost each commuter \$1491 per year. In aggregate, 62,136,000 gallons of fuel are wasted each year in Seattle due to traffic delays, contributing to pollution and climate change. The net cost of traffic to the city is estimated at \$3295 million annually. \cite{Schrank20152015Scorecard}.

Engineers can attempt to ameliorate traffic congestion by adding extra lanes to busy freeways; however, there is a growing incentive to consider how self-driving cars will impact traffic\cite{BusinessInsider102020}.  With an annual growth rate hovering at $2\%$, Seattle roadways are quickly becoming more crowded, and given the concentration of tech corporations in the region (such as Google), it is likely that the greater Seattle area will be the first to see a rise in the percentage of self-driving cars on the road.  We hope to understand how self-driving cars will affect traffic on Interstates 5, 90, and 405, as well as on State Route 520.  A successful model will provide us with an understanding of the impact of self-driving cars, which in turn will inform road design and policy changes to better facilitate transportation. 
