\section{Recommendations} \label{sec:recommendations}

\begin{itemize}
    \item Our result, most simply stated, is that the more self-driving cars there are, the better traffic will flow.  This being the case, our model suggests that policy should support the accessibility of safe, reliable self-driving cars to consumers. 
    \item Our model predicts that while the introduction of self-driving cars may eliminate minor traffic delays but, at best, will only reduce -- not eliminate -- major traffic delays.
    \item Our analysis suggests that for stretches of road at least three lanes wide, the designation of a lane for the exclusive use of self-driving cars will result in more efficient traffic flow during peak traffic hours when self-driving cars account for 5\% or more of vehicles on the road. The effectiveness of a designated lane holds so long as total traffic density is below $\sim$45\% which is above average peak traffic density.
    \item Analysis of our model does not reveal any circumstances under which the introduction of self-driving cars resulted in decreased traffic flow efficiency, so there is no evidence for a need to regulate self-driving cars on that account.
	\item The impacts of traffic delays are extremely costly across economic, environmental, and quality of life dimensions \cite{Schrank20152015Scorecard}. The impact of self-driving cars, as a potential boon to traffic flow through greater traffic efficiency but also as a potential threat by potentially increasing total traffic volume \cite{Org2014Www.vtpi.orgPlanning}, must be seriously and actively considered by policy makers.
\end{itemize}